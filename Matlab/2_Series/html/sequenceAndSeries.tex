
% This LaTeX was auto-generated from MATLAB code.
% To make changes, update the MATLAB code and republish this document.

\documentclass{article}
\usepackage{graphicx}
\usepackage{color}

\sloppy
\definecolor{lightgray}{gray}{0.5}
\setlength{\parindent}{0pt}

\begin{document}

    
    
\section*{Arithmetic series example}

\begin{par}
This is an example to implement an arithmetic series in Matlab code. This example was devloped during the summer school 2017 for Jade University in Wilhelmshaven by Gerardo M. Chavez-Campos, for more information visit \begin{verbatim}sagitario.itmorelia.edu.mx/gmarx/summerJade\end{verbatim}.
\end{par} \vspace{1em}

\subsection*{Contents}

\begin{itemize}
\setlength{\itemsep}{-1ex}
   \item Sequence and series definition
   \item Finite series
   \item Infinite series
   \item References
\end{itemize}


\subsection*{Sequence and series definition}

\begin{par}
A \textbf{Sequence} is a list of things (usually numbers) that are in order. A sequence is usually defined by a \textbf{Rule}, this is a way or equation to find each term [1]. Thus, in order to be able of determine ($u_n$, \$n\$th term) \textbf{the Rule} is written as a formula, where $n$ is any term.
\end{par} \vspace{1em}
\begin{par}
Now lets find a way to determine automatically each term ($u_n$) for the next sequence: 3, 5, 7, 9, ... and so forth,
\end{par} \vspace{1em}
\begin{verbatim}
for n=1:5
    u=2*n+1
end
\end{verbatim}

        \color{lightgray} \begin{verbatim}
u =

     3


u =

     5


u =

     7


u =

     9


u =

    11

\end{verbatim} \color{black}
    

\subsection*{Finite series}

\begin{par}
Now let ${u_n}$ be a sequence. Then the finite sum $S_n$ (partial sum) of $n_{th}$ order is:
\end{par} \vspace{1em}
\begin{par}
$$ S_n=u_1+u_2+u_3+...+u_n $$,
\end{par} \vspace{1em}
\begin{par}
and can be implemented in matlab as shown below.
\end{par} \vspace{1em}
\begin{verbatim}
for n=1:5
    Un(n)=2*n+1;
end
Sum=sum(Un)
\end{verbatim}

        \color{lightgray} \begin{verbatim}
Sum =

    35

\end{verbatim} \color{black}
    \begin{par}
Another way to calculate $S_n$ is with the following code:
\end{par} \vspace{1em}
\begin{verbatim}
Sum=0;
for n=1:5
    Un=2*n+1;
    Sum=Un+Sum
end
\end{verbatim}

        \color{lightgray} \begin{verbatim}
Sum =

     3


Sum =

     8


Sum =

    15


Sum =

    24


Sum =

    35

\end{verbatim} \color{black}
    

\subsection*{Infinite series}

\begin{par}

\begin{block}{Infinite Series}
    Let ${u_n}$ be a sequence. Then the Infinite sum order is:\\
    \begin{equation}
        \sum_{n=1}^\infty u_n=u_1+u_2+u_3+\hdots
    \end{equation}
  \end{block}

\end{par} \vspace{1em}
\begin{verbatim}
% Let ${u_n}$ be a sequence. Then the Infinite sum order is:
%
%    $$ sum_{n=1}^infty u_n=u_1+u_2+u_3+... $$
\end{verbatim}


\subsection*{References}

\begin{enumerate}
\setlength{\itemsep}{-1ex}
   \item \textbf{[1]} Math is Fun \begin{verbatim}www.mathisfun.com\end{verbatim}
   \item \begin{verbatim}sagitario.itmorelia.edu.mx/gmarx/summerJade\end{verbatim}
\end{enumerate}



\end{document}
    
